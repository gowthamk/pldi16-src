\section{Introduction}

Ruby-on-Rails, a web programming framework written in Ruby, has gained
wide adoption among Web 2.0 applications, such as Twitter, Airbnb,
Hulu, Groupon, and SoundCloud.  One of the factors that led to the
success of Rails framework is its singular focus on facilitating the
expression of complex web functionality via succint code. For
instance, Selfstarter~\cite{selfstarter}, a popular Rails application
that implements end-to-end functionality of a crowdfunding site,
including the integration with Amazon Payments, was implemented in
less than 600 lines of Ruby. While Rails's philosophy of prioritizing
``convention over configuration''~\cite{railsphilosophy} contributes
to its expressivity, the main facilitator, however, is the
meta-programming support offered by Ruby that enables rich
functionality, such as reflection, run-time code generation, singleton
objects, and also the much-maligned \emph{eval}~\cite{eval} function.  

Indeed, meta-programming features are available in many mainstream
languages~\cite{MetaOCaml, LMS}. However, their usage in mainstream
programming is not as prevalent as it is in Ruby. For example, the
idiomatic way to write a \C{Rectangle} class in Ruby is to rely on the
inbuilt \C{attr\_accessor} meta-function to introduce fields named
\C{length} and \C{breadth}, and define their setters and getters:
\begin{coderuby}
class Rectangle
  attr_accessor :length, :breadth
end
\end{coderuby}
Moreover, unlike the static languages, where meta-functions are
``evaluated away'' at the compile time, Ruby is unapolegetically
dynamic (no just-in-time compilation; all code is interpreted at
run-time), and makes no distinction between object-level code and
meta-level code. The dynamic nature of Ruby complicates the task of
building formal analyses for the language, which perhaps explains why
there are relatively few proposals for static analysis of Ruby code,
despite the language not being inherently safer than other dynamic
languages.

In this paper, we propose \name, a lightweight symbolic execution
framework for Ruby that also performs partial evaluation. \name
accepts a Ruby program, complete with meta-functions and their
applications, as its input, emits an equivalent Ruby
program\footnote{We do not yet have a formal proof for this claim.},
where all applications of meta-level functions are evaluated away. The
language of the generated program is a small subset of Ruby that is
amenable to static analysis. \name thus simplifies the task of
building static analyses for Ruby by preempting the need to reason
about meta features in an analysis-specific way. \name is lightweight
in the sense that it a language library (a Ruby ``gem'') rather than a
language implementation. \name was designed to coexist with an
unmodified Rails library, and rely on a concrete Ruby interpreter to
perform symbolic execution. We believe that this choice is appropriate
for a fast-evolving language like Ruby, where the semantics of the
language are only vaguely defined.

Chaudhuri and Foster~\cite{Rubyx} have previously proposed a symbolic
execution for Ruby-on-Rails tailor-made for their security analysis.
Their approach relies on a custom-built symbolic interpreter that
generates verification conditions. To the best of our knowledge, \name
is the first symbolic execution engine for Ruby-on-Rails that works
off an off-the-shelf interpreter, and generates Ruby code that can be
consumed by other analyses.

\section{Motivating Example}
\begin{figure}
  \begin{coderuby}
  class Micropost < ActiveRecord::Base
    belongs_to :user
    validates :content, presence: true, 
              length: { maximum: 140 }
    validates :user, presence: true
    # Some methods
    # ...
  end
  \end{coderuby}
\label{fig:postSave}
\caption{Micropost class (model) of Microblog Rails Application}
\end{figure}

\begin{figure}
  \begin{coderuby}
def save(post)
  v0 := post.id
  v1 := post.content
  v2 := post.user_id
  SQL (begin transaction)
  v6 := (v1.length)<=140
  if (v6==true) then
    v7 := SQL (SELECT  "users".* FROM "users"  
                WHERE "users"."id" = v2)
    v9 := v7.map do |v8|
      {id => v8.id, name => v8.name}
    end
    v10 = v9.first
    v11 = v10==nil
    if (v11==true) then
      SQL (rollback transaction)
    else
      v12 := SQL (INSERT INTO "microposts" VALUES (v1, v0, v2))
      SQL (commit transaction)
    end
  else
    SQL (rollback transaction)
  end
end
  \end{coderuby}

\label{fig:addPost}
\caption{Micropost class (model) of Microblog Rails Application}
\end{figure}

Fig.~\ref{fig:postSave} shows the implementation of \C{Micropost}
class taken from a Microblogging application developed using
Rails~\cite{sampleapp}. The class extends Rails's
\C{ActiveRecord::Base} class which implements Rails's
Object-Relational Mapping (ORM) system. Almost the entire
functionality of \C{Micropost} class was implemented using
meta-functions offered by \C{ActiveRecord::Base}. For example, instead
of implementing a custom logic to save a post to the database,
\C{Micropost} class relies on the generic \C{save} method provided
by\C{Base} class of \C{ActiveRecord}, but customizes it by specifying
additional constraints via the meta-function \C{validates}. The result
is a \C{save} method that performs additional checks (for eg., content
length $\le$ 140 characters) before saving the post to the database.
However, the customized \C{save} method never really exists; it's
logic is generated on-the-fly when the \C{save} method is called on a
concrete \C{Micropost} object. This makes it difficult for humans to
understand the program, and for the analyses to verify it.

\name runs the \C{save} method of \C{Micropost} on a symbolic input,
and generates its new definition as shown in Fig.~\ref{fig:addPost}
(simplified for clarity).  Observe that the generated definition has
no meta-function applications, is self-contained, and easier to
understand and analyze.

\section{Key Ideas Underlying \name}

% \begin{figure}
% \begin{coderuby}
% class SymbolicInteger < SymbolicValue
%   def +(other)
%     bin_op_exp = SymbolicAST::BinOp.new(self.ast,'+',other)
%     var = tracer.new_var_for(bool_op_exp)
%     return SymbolicInteger.new (var)
%   end
% end
% \end{coderuby}
% \label{fig:sym-int}
% \caption{Symbolic Integer in \name}
% \end{figure}

\name is based on some fundamental observations about Ruby in general,
and Ruby-on-Rails in particular. First, unlike in multi-stage
programming, Ruby has no stratification of the code that consumes
application's inputs and the code that consumes the application
itself. Under this context, symbolic execution is an effective partial
evaluation technique. Second, Ruby adopts \emph{duck typing}, meaning
that it does not distinguish between two objects that respond to same
messages (method calls), even when they are constructed from different
classes. This applies even to the objects of core types, such as
integers and strings. Consequently, a Ruby function expecting a
concrete integer input can instead be run on a symbolic integer as
long as the later behaves like the former\footnote{This is only true
as long as the control does not escape the boundary of Ruby and C, the
language in which certain core functionality is implemented.}. To
exploit this observation, \name framework defines symbolic
counterparts of all core classes that respond to the same methods as
core classes. However, instead of a concrete result, the symbolic
methods return new symbolic values, which may then flow into other
(symbolic or concrete) methods. \name assigns a unique symbolic AST
node to each symbolic value, and traces (i.e., logs to trace) the
relationship between new symbolic AST nodes and existing nodes. Third,
Ruby's reflection features allow control over class and method
bindings at the run-time (Even those of core types\footnote{\C{a+b} is
treated as \C{a.+(b)} in Ruby. This is a problem if \C{a} is concrete
and \C{b} is symbolic. To circumvent this problem, We override (at
run-time) the definition of concrete integer addition method so as to
treat \C{a+b} as \C{b+a}.}). Coupled with the fact that ``all data are
objects'' and ``all control are method applications'' in Ruby, the
semantics of any application can be controlled from the appliction
space itself. 

In \name, symbolic methods that are expected to return either \C{true}
or \C{false} do so by initially making an ambivalent choice, and later
backtracking to explore the other choice. For this purpose, we
implement an \C{amb} function similar to McCarthy's
\C{amb}~\cite{amb}, except that our \C{amb} relies on Unix processes
rather than continuations to checkpoint and restore the
state~\cite{StackOverflow}. Unix processes are needed because
restoring a stored continuation does not rollback side-effects on
heap. Whenever we make an ambivalent choice in method \C{f} of a
symbolic value \C{v}, we write an \C{if} expression to the trace (eg.,
\C{if (v.f() == true) then ...  end }).  The symbolic execution
continues in a new (child) process, tracing under the \C{then} branch,
while the parent process waits.  Once the child process returns, the
parent process resumes execution, assuming \C{v.f()} as \C{false}, and
traces under the \C{else} branch. While this approach does not
generalize to arbitrary loops, such loops are very rare in Ruby. The
idiomatic way to write a loop in Ruby is to use a higher-order
combinator, such as \C{map} and \C{each} on arrays, and \C{times} on
integer (for eg., \C{n.times {f()}} will execute \C{f()} \C{n} times).
Since higher-order combinators are methods, they too can be traced and
made to return a symbolic value, just like other methods. 

The result of symbolic execution is a trace that contains a slice
of the original program that is data-dependent on its inputs. Any
interactions with the environment (eg., database I/O) are also traced,
given that such interactions happen via a symbolic adapter. 

\section{Implementation and Experiments}

\name is implemented as a Ruby library that works on top of an
off-the-shelf Ruby interpreter (We used MRI~\cite{MRI} for
development). From an engineering perspective, leveraging an
off-the-shelf Ruby interpreter and relying on Unix processes for heap
checkpointing and restoration enabled an implementation comprising
rougly only 1300 lines of Ruby code, packaged as a Ruby
gem\footnote{Source code available at
\url{https://github.com/gowthamk/conflict_analysis}}.

We used \name to compute summaries of various operations supported by
applications similar to the one described in \S~\ref{sec:motivation}.
Although the complexity being exponential in terms on number of
branches, \name was able to compute method summaries within 2s. 

It is conjectured that the Ruby code generated by name runs faster
than the original code that involves calls to meta-functions. We have
not tested this hypothesis, and we plan to explore this direction in
the near future.

